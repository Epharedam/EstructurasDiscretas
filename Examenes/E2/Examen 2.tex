\documentclass[a4paper,10pt]{article}
\usepackage[utf8]{inputenc}
\usepackage[spanish]{babel}
%\usepackage{geometry}
%\addtolength{\hoffset}{-0.2cm}
\usepackage{hyperref}
\usepackage{amsfonts}
%Multicolumns
\usepackage{multicol}
\usepackage{fdsymbol}
\usepackage{tabularx}
\usepackage{graphicx}
\usepackage{xcolor}
\usepackage{float}
\usepackage{booktabs}
%\usepackage[ruled, lined, linesnumbered, commentsnumbered, longend]{algorithm2e}
\usepackage{algpseudocode}
\usepackage{marginnote}
\usepackage{enumitem}
\reversemarginpar
\pagenumbering{gobble}



\newcolumntype{A}{ >{$} r <{$} @{} >{${}} l <{$} } 


\title{Estructuras Discretas\\Examen 2}


\begin{document}

\maketitle

\vspace{-15pt}
\noindent \textbf{Resuelve de manera limpia y ordenada los siguientes ejercicios. 
Indica claramente el n\'umero de pregunta que se esta resolviendo.}


\begin{enumerate}	
		
	\item \marginnote{{\em 2 puntos}}
		Decide, utilizando interpretaciones o \textit{tableaux}, si el siguiente conjunto es satisfacible. En caso de serlo, da un modelo para el conjunto. 
		$$\Gamma = \{ \neg (p \to q), \neg(q \land r), r \leftrightarrow \neg p, \neg q \to ( p \land \neg r) \}$$
1. \\
   - \(p \rightarrow q\) es equivalente a \(\neg p \vee q\)\\
   - \(r \leftrightarrow \neg p\) es equivalente a \((r \wedge \neg p) \vee (\neg r \wedge p)\)\\
   - \(\neg q \rightarrow (p \wedge \neg r)\) es equivalente a \(q \vee (p \wedge \neg r)\)\\
   Así:\\
   \[
   \Gamma=\{\neg(\neg p \vee q), \neg(q \wedge r), (r \wedge \neg p) \vee (\neg r \wedge p), q \vee (p \wedge \neg r)\}
   \] \\
2.\\
   - \(\neg\neg A\), la reescribimos \(A\).\\
   - \(A \wedge B\), la reescribimos \(A\) y \(B\)\\
   - \(A \vee B\), la reescribimos \(A\) y \(B\) \\
   - \(\neg(A \wedge B)\), la reescribimos \(\neg A\) y \(\neg B\) \\
   - \(\neg(A \vee B)\), la reescribimos \(\neg A\) y \(\neg B\)\\
3.\\
   \[
   \begin{array}{cccc}
   & & p \wedge \neg q & \\
   & \swarrow & & \searrow \\
   p & & \neg q & \\
   & & \neg q \vee \neg r & \\
   & \swarrow & & \searrow \\
   \neg q & & \neg r & \\
   & & r \wedge \neg p \vee \neg r \wedge p & \\
   & \swarrow & & \searrow \\
   r \wedge \neg p & & \neg r \wedge p & \\
   & \swarrow & & \searrow \\
   r & \neg p & \neg r & p \\
   \end{array}
   \]

3. La rama izq. contiene \(p\) y \(\neg p\), (!)\\
   - La derecha \(r\) y \(\neg r\), (!)\\

   $\therefore$ no es satisfacible

  
	\item \marginnote{{\em 2 puntos}}
		Usando interpretaciones o \textit{tableaux}, determina si el siguiente argumento es correcto.   En caso de no serlo exhibe una interpretaci\'on que haga verdaderas a las premisas y falsa a la conclusi\'on.
		$$(r \lor u) \to s, r, s \to t  / \therefore  \  t \lor u.$$\\
\begin{multicols}{2}
Premisas:\\
1. $(r \lor u) \to s$\\
2. $r$\\
3. $s \to t$\\

Conclusion:\\
$t \lor u$\\
\end{multicols}

\begin{multicols}{2}
1. \\
   - $(r \lor u) \to s$\\
   - $r$\\
   - $s \to t$\\
   - $\neg (t \lor u)$\\
   
2. \\
   - $A \to B$, agregamos $\neg A$ y $B$\\
   - $\neg (A \lor B)$, agregamos $\neg A$ yand $\neg B$\\
   - $\neg (A \land B)$, agregamos $\neg A$ o $\neg B$\\
\end{multicols}
\begin{multicols}{2}
3. Tenemos\\
   - $(r \lor u) \to s$, agregamos\\
   $\neg (r \lor u)$ y $s$\\
   - $r$, agregamos $r$\\
   - $s \to t$, agregamos $\neg s$ y $t$\\
   - $\neg (t \lor u)$, agregamos $\neg t$ y $\neg u$\\
\newline
\newline
\newline
\newline
4. Así\\
   - $\neg (r \lor u)$\\
   - $s$\\
   - $r$\\
   - $\neg s$\\
   - $t$\\
   - $\neg t$\\
   - $\neg u$\\
\end{multicols}
5. Tiene $s$, $\neg s$, $t$ y $\neg t$. Hay una contradicción (!)\\
$\therefore$ el argumento es válido\\
		
	\item \marginnote{{\em 4 puntos}}
		Traduce el siguiente argumento a lenguaje formal y demuestra que
		es correcto usando derivaciones. Justifica la obtención de la expresión mostrada en cada paso: indica si es una premisa, una suposición, resultado de aplicar una regla de inferencia en una o más líneas anteriores (por ejemplo, MP 1, 2 para indicar obtención por medio de Modus Ponens con las líneas 1 y 2), o razomamiento ecuacional (RE).
		
		\vspace{5pt}
		
		\textit{Si Chubaka no es perro, entonces no es cierto que sea alado o que sea borogove. Si Chubaka es quelite, entonces es alado. Sabemos que Chubaka no es perro. Luego entonces, Chubaka no es quelite.}
    Let's denote the following:
\begin{multicols}{2}
    P: Chubaka es perro. \\
    Q: Chubaka es alado.\\
    R: Chubaka es borogove.\\
    S: Chubaka es quelite.\\

    Así\\
    1. $\neg$P $\rightarrow$ $(\neg Q \land \neg R)$ (Premisa)\\
    2. S $\rightarrow$ Q (Premisa)\\
    3. $\neg$ P (Premisa)\\
    4.$\therefore$ ¬S
\end{multicols}

Usamos Modus Ponens (MP) en 1 y 3:\\
4. $\neg Q \land \neg R$ (MP 1, 3)\\
Simplificación\\
5. $\neg$Q (Simplificacion 4)\\
Modus Tollens en 2 y 5:\\
6. $\neg$S (MT 2, 5)\\
$\therefore$ Chubaka no es quelite\\

	\vspace{5pt}
		
	\item \marginnote{{\em 2 puntos}}
		Construye la siguiente derivación. Justifica el proceso como en la pregunta anterior.\\
		$$\vdash \, (\neg p \, \land q) \lor (p \land \neg \, q) \to (\neg \, p \land (\neg \, p \land q)) \lor (p \land (p \land \neg \, q))$$\\
%%%%%%%%%%%%%%
1. $\neg p \land q$ (Suposición)\\
2. $p \land \neg q$ (Suposición)\\
3. $\neg p \land (\neg p \land q)$ (De 1, Conjuncion)\\
4. $p \land (p \land \neg q)$ (De 2, Conjuncion)\\
5. $(\neg p \land q) \lor (p \land \neg q)$ (De 1 y 2, Disyuncion)\\
6. $(\neg p \land (\neg p \land q)) \lor (p \land (p \land \neg q))$ (De 3 y 4, Disyuncion)\\
7. $(\neg p \land q) \lor (p \land \neg q) \to (\neg p \land (\neg p \land q)) \lor (p \land (p \land \neg q))$ (De 5 y 6, Condicional)\\

$\therefore$ el argumento es correcto

		
\end{enumerate}


\end{document}
